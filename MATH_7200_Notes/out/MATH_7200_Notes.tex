\documentclass{article}
% generated by Madoko, version 1.0.1
%mdk-data-line={1}


\usepackage[heading-base={2},section-num={False},bib-label={True}]{madoko2}
\usepackage{bm}
\usepackage{physics}
\usepackage{mathtools}


\begin{document}



%mdk-data-line={16}
\newcommand{\R}{\mathbb{R}}
\renewcommand{\C}{\mathbb{C}}
\newcommand{\Z}{\mathbb{Z}}
\newcommand{\id}{\operatorname{id}}
\newcommand{\im}{\operatorname{im}}
\newcommand{\pt}{\operatorname{pt}}
%mdk-data-line={24}
\mdxtitleblockstart{}
%mdk-data-line={24}
\mdxtitle{\mdline{24}MATH 7200 Notes
    Algebraic Topology}%mdk

%mdk-data-line={27}
\mdxsubtitle{\mdline{27}as taught by Prof. Anna-Marie Bohmann
    Vanderbilt University
    Fall Semester 2016}%mdk
\mdxauthorstart{}
%mdk-data-line={32}
\mdxauthorname{\mdline{32}Transcribed by David K. Zhang}%mdk
\mdxauthorend\mdtitleauthorrunning{}{}\mdxtitleblockend%mdk

%mdk-data-line={26}
\section{\mdline{26}1.\hspace*{0.5em}\mdline{26}Lecture 1 (2016-08-25)}\label{sec-lecture-1-2016-08-25}%mdk%mdk

%mdk-data-line={28}
\noindent\mdline{28}\emph{(Discussion of syllabus omitted.)}\mdline{28}%mdk

%mdk-data-line={30}
\mdline{30}The goal of algebraic topology is to study topological spaces via their algebraic invariants. A familiar example from previous courses is the fundamental group, which associates to each topological space \mdline{30}$X$\mdline{30} and point \mdline{30}$x_0 \in X$\mdline{30} a group \mdline{30}$\pi_1(X, x_0).$\mdline{30}%mdk

%mdk-data-line={32}
\mdline{32}In this course, we will also study homology and cohomology groups  (chapters 2-3 of Hatcher\mdline{32}'\mdline{32}s text), which are relatively easy to compute, in addition to higher homotopy groups (chapter 4), which are more powerful but harder to compute.%mdk

%mdk-data-line={34}
\subsection{\mdline{34}1.1.\hspace*{0.5em}\mdline{34}Review of Essential Concepts}\label{sec-review-of-essential-concepts}%mdk%mdk

%mdk-data-line={36}
\noindent\mdline{36}We want to study topological spaces up to homotopy. (Henceforth all spaces will be assumed topological, and all maps assumed continuous.) We say that two maps \mdline{36}$f_0, f_1: X \to Y$\mdline{36} are \mdline{36}\textbf{\emph{homotopic}}\mdline{36}, written \mdline{36}$f_0 \simeq f_1,$\mdline{36} if there exists a map \mdline{36}$H: X \times I \to Y$\mdline{36} such that \mdline{36}$H(x,i) = f_i(x)$\mdline{36} for \mdline{36}$i = 0,1.$\mdline{36}%mdk

%mdk-data-line={38}
\mdline{38}A map \mdline{38}$f: X \to Y$\mdline{38} is a \mdline{38}\textbf{\emph{homotopy equivalence}}\mdline{38} if there exists a map \mdline{38}$g: Y \to X$\mdline{38} such that \mdline{38}$fg = \id_Y$\mdline{38} and \mdline{38}$gf = \id_X.$\mdline{38} If this occurs, we say that \mdline{38}$X$\mdline{38} and \mdline{38}$Y$\mdline{38} are \mdline{38}\textbf{\emph{homotopy equivalent}}\mdline{38}, or simply \mdline{38}\textbf{\emph{equivalent}}\mdline{38} if no confusion can arise. For example, the letters A and D are homotopy equivalent, but A and B are not. Homotopy equivalence is indeed an equivalence relation\mdline{38} \mdline{38}\textemdash{}\mdline{38} the meaning of studying a space \mdline{38}\textquotedblleft{}up to homotopy\textquotedblright{}\mdline{38} is to study its equivalence class under homotopy equivalence.%mdk

%mdk-data-line={40}
\mdline{40}A space with the homotopy equivalence type of a point is called \mdline{40}\textbf{\emph{contractible}}\mdline{40}. For example, \mdline{40}$\R^n$\mdline{40} is contractible. (The homotopy is given by multiplication by \mdline{40}$t.$\mdline{40}) In general, we won\mdline{40}'\mdline{40}t necessarily write down our homotopies explicitly, as long as we can describe them sufficiently clearly.%mdk

%mdk-data-line={42}
\mdline{42}A map which is homotopic to the \mdline{42}\textquotedblleft{}squish everything to a point\textquotedblright{}\mdline{42} map is called \mdline{42}\textbf{\emph{nullhomotopic}}\mdline{42}. Occasionally the word \mdline{42}\textbf{\emph{essential}}\mdline{42} is used to mean \mdline{42}\textquotedblleft{}not nullhomotopic.\textquotedblright{}\mdline{42}%mdk

%mdk-data-line={44}
\mdline{44}For the purposes of this class, we will restrict our attention to \mdline{44}\textquotedblleft{}nice\textquotedblright{}\mdline{44} spaces, where the word \mdline{44}\textquotedblleft{}nice\textquotedblright{}\mdline{44} means \mdline{44}\textquotedblleft{}sufficiently well-behaved to make our theorems work.\textquotedblright{}\mdline{44} (This typically includes, for example, the Hausdorff property.) Examples include \mdline{44}\textbf{\emph{manifolds}}\mdline{44} and \mdline{44}\textbf{\emph{CW-complexes}}\mdline{44}.%mdk

%mdk-data-line={46}
\mdline{46}A CW-complex is a space that is \mdline{46}\textquotedblleft{}inductively constructed\textquotedblright{}\mdline{46} by attaching cells (i.e. disks of various dimensions). For example, by gluing together two points, two segments, and a disk, we obtain a (closed) filled circle. In general, we start with a discrete space (collection of points) \mdline{46}$X^0$\mdline{46} and inductively form a space \mdline{46}$X^n,$\mdline{46} called the \mdline{46}\textbf{\emph{$\bm{n}$-skeleton}}\mdline{46}, from \mdline{46}$X^{n-1}$\mdline{46} by attaching \mdline{46}$n$\mdline{46}-cells \mdline{46}$\{e_\alpha^n\}_{\alpha \in A}$\mdline{46} along maps \mdline{46}$\phi_\alpha: S_\alpha^{n-1} \to X^{n-1}.$\mdline{46} Explicitly, \mdline{46}$X^n$\mdline{46} is the quotient space%mdk
\noindent\noindent\[%mdk-data-line={48}
\frac{\qty(\bigsqcup_{\alpha \in A} e_\alpha^n) \sqcup X^{n-1}}{x \sim \phi_\alpha(x)}
\]%mdk
\noindent\mdline{50}where the denominator indicates that \mdline{50}$x$\mdline{50} is identified with \mdline{50}$\phi_\alpha(x)$\mdline{50} for all \mdline{50}$x \in \partial e_\alpha^n.$\mdline{50}

%mdk-data-line={52}
\mdline{52}\emph{(Some examples of CW-complexes omitted.)}\mdline{52}%mdk

%mdk-data-line={54}
\subsection{\mdline{54}1.2.\hspace*{0.5em}\mdline{54}Operations on Spaces}\label{sec-operations-on-spaces}%mdk%mdk

%mdk-data-line={56}
\noindent\mdline{56}The product or disjoint union of two CW-complexes is another CW-complex. Moreover, if \mdline{56}$X$\mdline{56} is a CW-complex, and \mdline{56}$A \subseteq X$\mdline{56} is a \mdline{56}\textbf{\emph{subcomplex}}\mdline{56} (closed subspace which is a union of cells), then we can construct the \mdline{56}\textbf{\emph{quotient space}}\mdline{56} \mdline{56}$X/A$\mdline{56} which bears an induced CW-structure. (Everything in \mdline{56}$A$\mdline{56} gets crushed to a point, and cells attached to \mdline{56}$A$\mdline{56} become attached to that point.)%mdk

%mdk-data-line={58}
\mdline{58}A less familiar operation is that of \mdline{58}\textbf{\emph{suspension}}\mdline{58}. If \mdline{58}$X$\mdline{58} is a space, the \mdline{58}\textbf{\emph{cone on $\bm{X}$}}\mdline{58} is the space \mdline{58}$CX$\mdline{58} given by \mdline{58}$(X \times I) / (X \times \{0\} \sim \pt).$\mdline{58} (Imagine constructing a \mdline{58}\textquotedblleft{}cylinder\textquotedblright{}\mdline{58} \mdline{58}$X \times I$\mdline{58} and crushing the bottom face to a point.) The suspension \mdline{58}$SX$\mdline{58} is given by \mdline{58}$(X \times I) / (X \times \{0\} \sim \pt_1,\ X \times \{1\} \sim \pt_2)$\mdline{58} with both faces crushed in. If \mdline{58}$f: X \to Y$\mdline{58} is a map, its \mdline{58}\textbf{\emph{mapping cone}}\mdline{58} \mdline{58}$M_f$\mdline{58} is \mdline{58}$(CX \sqcup Y) / ((x,1) \sim f(x)).$\mdline{58} Think of a witch\mdline{58}'\mdline{58}s hat with the \mdline{58}\textquotedblleft{}brim\textquotedblright{}\mdline{58} as \mdline{58}$Y$\mdline{58} and the \mdline{58}\textquotedblleft{}cone\textquotedblright{}\mdline{58} as \mdline{58}$CX,$\mdline{58} with the cone attached to the brim via \mdline{58}$f.$\mdline{58}%mdk

%mdk-data-line={62}
\section{\mdline{62}2.\hspace*{0.5em}\mdline{62}Lecture 2 (2016-08-30)}\label{sec-lecture-2-2016-08-30}%mdk%mdk

%mdk-data-line={64}
\noindent\mdline{64}Recall that the fundamental group \mdline{64}$\pi_1(X)$\mdline{64} depends only on low-dimensional information. In fact, for CW-complexes, it is uniquely determined by the 0, 1, and 2-skeleta. (We won\mdline{64}'\mdline{64}t prove this, but it isn\mdline{64}'\mdline{64}t hard to see intuitively.) This means that \mdline{64}$\pi_1$\mdline{64} cannot distinguish between, say, \mdline{64}$S^3$\mdline{64} and \mdline{64}$S^{1,000,000}.$\mdline{64}%mdk

%mdk-data-line={66}
\mdline{66}We would obviously like to study algebraic invariants of topological spaces that take higher-dimensional information into account. One obvious generalization is the \mdline{66}\textbf{\emph{higher homotopy groups}}\mdline{66}, where instead of considering homotopy classes of maps \mdline{66}$S^1 \to X,$\mdline{66} we instead look at homotopy classes of maps \mdline{66}$S^n \to X.$\mdline{66} This is a fine idea, but these groups turn out to be \mdline{66}\emph{very}\mdline{66} difficult to compute.%mdk

%mdk-data-line={68}
\mdline{68}Another solution is \mdline{68}\textbf{\emph{homology}}\mdline{68}. \mdline{68}\emph{(Motivating example from Hatcher's text omitted. See pg. 99.)}\mdline{68}%mdk

%mdk-data-line={70}
\mdline{70}We first study \mdline{70}\textbf{\emph{simplicial homology}}\mdline{70}. Our building blocks in this context are the \mdline{70}\textbf{\emph{$\bm{n}$-simplices}}\mdline{70} \mdline{70}$\Delta^n,$\mdline{70} the \mdline{70}$n$\mdline{70}-dimensional analogues of the tetrahedron. We can regard \mdline{70}$\Delta^n$\mdline{70} as the smallest convex set in \mdline{70}$\R^m$\mdline{70} (for some \mdline{70}$m$\mdline{70}) containing \mdline{70}$n+1$\mdline{70} affinely independent points. In particular, we have the \mdline{70}\textbf{\emph{standard $\bm{n}$-simplex}}\mdline{70}%mdk
\noindent\noindent\[%mdk-data-line={72}
\Delta^n = \qty{(t_0, t_1, \dots, t_n) \in \R^{n+1} :
  t_i \ge 0\ \forall i \text{ and } \sum_{i=0}^n t_i = 1}.
\]%mdk
\noindent\mdline{75}We will include an ordering on the vertices as part of the information encoded by a simplex. Given an \mdline{75}$n$\mdline{75}-simplex \mdline{75}$\Delta^n,$\mdline{75} we can throw out the \mdline{75}$i$\mdline{75}th vertex to obtain an \mdline{75}$(n-1)$\mdline{75}-simplex, called the \mdline{75}\textbf{\emph{$\bm{i}$th face}}\mdline{75} of \mdline{75}$\Delta^n.$\mdline{75} An ordering of vertices on the face is inherited from the ordering on \mdline{75}$\Delta^n.$\mdline{75} The union of all faces of \mdline{75}$\Delta^n$\mdline{75} is the \mdline{75}\textbf{\emph{boundary}}\mdline{75} \mdline{75}$\partial \Delta^n.$\mdline{75}

%mdk-data-line={77}
\mdline{77}Given a space \mdline{77}$X,$\mdline{77} a \mdline{77}\textbf{\emph{$\bm{\Delta}$-complex structure}}\mdline{77} on \mdline{77}$X$\mdline{77} is a collection of maps \mdline{77}$\{\sigma_\alpha: \Delta^n \to X\}$\mdline{77} (where \mdline{77}$n$\mdline{77} depends on \mdline{77}$\alpha$\mdline{77}) such that%mdk

%mdk-data-line={79}
\begin{enumerate}%mdk

%mdk-data-line={79}
\item{}
%mdk-data-line={79}
\mdline{79}The restriction of each \mdline{79}$\sigma_\alpha$\mdline{79} to \mdline{79}$\operatorname{int}(\Delta^n) = \Delta^n \setminus \partial \Delta^n$\mdline{79} is injective, and each point \mdline{79}$x \in X$\mdline{79} is in the image of exactly one such restriction.%mdk%mdk

%mdk-data-line={81}
\item{}
%mdk-data-line={81}
\mdline{81}The restriction of each \mdline{81}$\sigma_\alpha$\mdline{81} to a face of \mdline{81}$\Delta^n$\mdline{81} is another map \mdline{81}$\sigma_\beta: \Delta^{n-1} \to X$\mdline{81} in our collection.%mdk%mdk

%mdk-data-line={83}
\item{}
%mdk-data-line={83}
\mdline{83}$A \subseteq X$\mdline{83} is open iff \mdline{83}$\sigma_\alpha^{-1}(A)$\mdline{83} is open in \mdline{83}$\Delta^n$\mdline{83} for each \mdline{83}$\alpha.$\mdline{83}%mdk%mdk
%mdk
\end{enumerate}%mdk

%mdk-data-line={85}
\noindent\mdline{85}This essentially amounts to saying that \mdline{85}$X$\mdline{85} is a bunch of simplices glued together along faces. \mdline{85}$\Delta$\mdline{85}-complex structures are precisely what we need to define simplicial homology.%mdk

%mdk-data-line={87}
\mdline{87}\textbf{Definition:}\mdline{87} Given a space \mdline{87}$X$\mdline{87} and a \mdline{87}$\Delta$\mdline{87}-complex structure on \mdline{87}$X,$\mdline{87} let \mdline{87}$\Delta_n(X)$\mdline{87} be the free abelian group on the set \mdline{87}$\{\sigma_\alpha: \Delta^n \to X\}$\mdline{87} of \mdline{87}$n$\mdline{87}-simplices of \mdline{87}$X.$\mdline{87} This is the group of finite formal linear combinations of the maps \mdline{87}$\sigma_\alpha,$\mdline{87} thought of as formal symbols. We then define the boundary homomorphisms \mdline{87}$\partial_n: \Delta_n(X) \to \Delta_{n-1}(X)$\mdline{87} as follows:%mdk
\noindent\noindent\[%mdk-data-line={89}
\partial_n(\sigma_\alpha) = \sum_{i=0}^n (-1)^i
  \eval{\sigma_\alpha}_{[v_0, \dots, \hat{v}_i, \dots, v_n]}
\]%mdk
\noindent\mdline{92}Here, the hat means that \mdline{92}$v_i$\mdline{92} is omitted. This means that we have a sequence
\noindent\noindent\[%mdk-data-line={94}
\Delta_{n+1}(X) \overset{\partial_{n+1}}{\longrightarrow}
\Delta_{n}(X) \overset{\partial_{n}}{\longrightarrow}
\Delta_{n-1}(X) \overset{\partial_{n+1}}{\longrightarrow}
\cdots \overset{\partial_{1}}{\longrightarrow} \Delta_0(X).
\]%mdk
\noindent\mdline{100}\textbf{Algebraic Interlude:}\mdline{100} A \mdline{100}\textbf{\emph{chain complex}}\mdline{100} \mdline{100}$A_*$\mdline{100} is a sequence of (free) Abelian groups linked by homomorphisms \mdline{100}$d_n: A_n \to A_{n-1}$\mdline{100} such that \mdline{100}$d^2 = 0.$\mdline{100} That is, \mdline{100}$d_n \circ d_{n+1} = 0$\mdline{100} for all \mdline{100}$n.$\mdline{100} Given a chain complex \mdline{100}$A_*,$\mdline{100} its \mdline{100}\textbf{\emph{homology}}\mdline{100} is a graded abelian group \mdline{100}$H_*(A)$\mdline{100} such that \mdline{100}$H_n(A) = \ker d_n / \im d_{n+1}.$\mdline{100} We call elements of \mdline{100}$A_*$\mdline{100} \mdline{100}\textbf{\emph{chains}}\mdline{100}, elements of \mdline{100}$\ker d_n$\mdline{100} \mdline{100}\textbf{\emph{cycles}}\mdline{100}, and elements of \mdline{100}$\im d_{n+1}$\mdline{100} \mdline{100}\textbf{\emph{boundaries}}\mdline{100}. Elements of \mdline{100}$H_*(A)$\mdline{100} are called \mdline{100}\textbf{\emph{homology classes}}\mdline{100}.

%mdk-data-line={102}
\mdline{102}\emph{(Standard proof of $\partial^2 = 0$ omitted.)}\mdline{102}%mdk

%mdk-data-line={104}
\mdline{104}This shows that \mdline{104}$\Delta_*(X)$\mdline{104} is a chain complex. Its homology \mdline{104}$H_*(\Delta_*(X))$\mdline{104} is the \mdline{104}\textbf{\emph{simplicial homology}}\mdline{104} of \mdline{104}$X.$\mdline{104} We denote the groups of this homology by \mdline{104}$H_n^\Delta(X).$\mdline{104} At this point, we make some observations:%mdk

%mdk-data-line={106}
\begin{enumerate}%mdk

%mdk-data-line={106}
\item{}
%mdk-data-line={106}
\mdline{106}It is not at all clear that the simplicial homology groups are invariants of the underlying space \mdline{106}$X.$\mdline{106} A priori, it\mdline{106}'\mdline{106}s perfectly possible that two different \mdline{106}$\Delta$\mdline{106}-complex structures on \mdline{106}$X$\mdline{106} might give different simplicial homologies.%mdk%mdk

%mdk-data-line={108}
\item{}
%mdk-data-line={108}
\mdline{108}If \mdline{108}$X$\mdline{108} has finitely many \mdline{108}$n$\mdline{108}-simplices in its \mdline{108}$\Delta$\mdline{108}-complex structure, then \mdline{108}$H_n^\Delta(X)$\mdline{108} is finitely generated.%mdk%mdk

%mdk-data-line={110}
\item{}
%mdk-data-line={110}
\mdline{110}If \mdline{110}$X$\mdline{110} is finite-dimensional, then \mdline{110}$H_k^\Delta(X) = 0$\mdline{110} for \mdline{110}$k > n.$\mdline{110}%mdk%mdk
%mdk
\end{enumerate}%mdk

%mdk-data-line={114}
\section{\mdline{114}3.\hspace*{0.5em}\mdline{114}Lecture 3 (2016-09-01)}\label{sec-lecture-3-2016-09-01}%mdk%mdk

%mdk-data-line={116}
\subsection{\mdline{116}3.1.\hspace*{0.5em}\mdline{116}Singular Homology}\label{sec-singular-homology}%mdk%mdk

%mdk-data-line={118}
\noindent\mdline{118}A \mdline{118}\textbf{\emph{singular $\bm{n}$-simplex}}\mdline{118} in a space \mdline{118}$X$\mdline{118} is a map \mdline{118}$\sigma: \Delta^n \to X.$\mdline{118} Let \mdline{118}$C_n(X)$\mdline{118} be the free abelian group on the set of singular \mdline{118}$n$\mdline{118}-simplices in \mdline{118}$X.$\mdline{118} We get a chain complex%mdk
\noindent\noindent\[%mdk-data-line={120}
\cdots \longrightarrow C_{n+1}(X)
\overset{\partial_{n+1}}{\longrightarrow} C_n(X)
\overset{\partial_{n}}{\longrightarrow} C_{n-1}(X)
\longrightarrow \cdots
\]%mdk
\noindent\mdline{125}with boundary maps
\noindent\noindent\[%mdk-data-line={127}
\partial_n(\sigma) = \sum_{i=0}^n (-1)^i
\eval{\sigma}_{[v_0, \dots, \hat{v}_i, \dots, v_n]}.
\]%mdk
\noindent\mdline{130}Note that \mdline{130}$\eval{\sigma}_{[v_0, \dots, \hat{v}_i, \dots, v_n]}$\mdline{130} is a map from \mdline{130}$\Delta^{n-1}$\mdline{130} to \mdline{130}$X.$\mdline{130} The same proof as before gives \mdline{130}$\partial^2 = 0.$\mdline{130}

%mdk-data-line={132}
\mdline{132}\textbf{Definition:}\mdline{132} The \mdline{132}\textbf{\emph{$\bm{n}$th singular homology group}}\mdline{132} of \mdline{132}$X,$\mdline{132} denoted by \mdline{132}$H_n(X),$\mdline{132} is the \mdline{132}$n$\mdline{132}th homology group of this chain complex.%mdk
\noindent\noindent\[%mdk-data-line={134}
H_n(X) = \ker \partial_n / \im \partial_{n+1}.
\]%mdk
\noindent\mdline{137}One immediately observes that if \mdline{137}$X$\mdline{137} is homeomorphic than \mdline{137}$X',$\mdline{137} then \mdline{137}$H_n(X) \equiv H_n(X').$\mdline{137} However, it\mdline{137}'\mdline{137}s unclear that these groups would ever be finitely generated.

%mdk-data-line={139}
\mdline{139}\textbf{Remark:}\mdline{139} We can, in fact, regard singular homology as a special case of simplicial homology by building \mdline{139}$S(X)$\mdline{139} as a \mdline{139}$\Delta$\mdline{139}-complex with one \mdline{139}$n$\mdline{139}-simplex for each singular \mdline{139}$n$\mdline{139}-simplex in \mdline{139}$X,$\mdline{139} with each \mdline{139}$n$\mdline{139}-simplex attached to its faces as \mdline{139}$(n-1)$\mdline{139}-simplices in the natural way. This gives a giant \mdline{139}$\Delta$\mdline{139}-complex \mdline{139}$S(X)$\mdline{139} which clearly satisfies \mdline{139}$H_n^\Delta(S(X)) = H_n(X).$\mdline{139}%mdk

%mdk-data-line={141}
\subsection{\mdline{141}3.2.\hspace*{0.5em}\mdline{141}Geometric interpretation of singular homology}\label{sec-geometric-interpretation-of-singular-homology}%mdk%mdk

%mdk-data-line={143}
\noindent\mdline{143}In general, a singular \mdline{143}$n$\mdline{143}-chain \mdline{143}$\xi$\mdline{143} can be written as a sum \mdline{143}$\xi = \sum_i \epsilon_i \sigma_i$\mdline{143} with \mdline{143}$\epsilon_i = \pm 1$\mdline{143} (and some simplices \mdline{143}$\sigma_i$\mdline{143} possibly repeated.) \mdline{143}$\partial\xi$\mdline{143} is then a linear combination of \mdline{143}$(n-1)$\mdline{143}-simplices with sign \mdline{143}$\pm 1.$\mdline{143}%mdk

%mdk-data-line={145}
\mdline{145}\emph{(Some things happened here that I don't really understand. See pgs. 108-109 of Hatcher's text. The upshot is that $H_1(X)$ is represented by maps $S^1 \to X,$ and $H_2(X)$ is represented by maps of oriented surfaces into $X.$)}\mdline{145}%mdk

%mdk-data-line={147}
\subsection{\mdline{147}3.3.\hspace*{0.5em}\mdline{147}Properties of singular homology}\label{sec-properties-of-singular-homology}%mdk%mdk

%mdk-data-line={149}
\noindent\mdline{149}\textbf{Proposition:}\mdline{149} If \mdline{149}$X$\mdline{149} decomposes into path components as \mdline{149}$X = \bigsqcup_\alpha X_\alpha,$\mdline{149} then \mdline{149}$H_n(X) = \bigoplus_\alpha H_n(X_\alpha).$\mdline{149}%mdk

%mdk-data-line={151}
\mdline{151}\emph{Proof:}\mdline{151} Since \mdline{151}$\Delta^n$\mdline{151} is path-connected, its continuous image is path-connected in \mdline{151}$X,$\mdline{151} and hence lies in some \mdline{151}$X_\alpha.$\mdline{151} This means that the chain groups decompose as \mdline{151}$C_n(X) = \bigoplus_\alpha C_n(X_\alpha).$\mdline{151} The boundary maps respect this decomposition, and hence the homology groups split as desired. \mdline{151}\textbf{QED}\mdline{151}%mdk

%mdk-data-line={153}
\mdline{153}\textbf{Proposition:}\mdline{153} If \mdline{153}$X$\mdline{153} is nonempty and path-connected, then \mdline{153}$H_0(X) = \mathbb{Z}.$\mdline{153} For any nonempty \mdline{153}$X,$\mdline{153} \mdline{153}$H_0(X)$\mdline{153} is a direct sum of copies of \mdline{153}$\mathbb{Z},$\mdline{153} one for each path component.%mdk

%mdk-data-line={155}
\mdline{155}\emph{Proof:}\mdline{155} By definition, \mdline{155}$H_0(X) = C_0(X) / \im \partial_1.$\mdline{155} Define a map \mdline{155}$\epsilon: C_0(X) \to \mathbb{Z}$\mdline{155} which sends a linear combination of \mdline{155}$0$\mdline{155}-simplices (i.e. points of \mdline{155}$X$\mdline{155}) to the sum of its coefficients. This is indeed an abelian group homomorphism, and if \mdline{155}$X$\mdline{155} is nonempty, then \mdline{155}$\epsilon$\mdline{155} is surjective.%mdk

%mdk-data-line={157}
\mdline{157}We claim that \mdline{157}$\ker \epsilon = \im \partial_1$\mdline{157} if \mdline{157}$X$\mdline{157} is path-connected. The inclusion \mdline{157}$\im \partial_1 \subseteq \ker \epsilon$\mdline{157} is clear from the definition of \mdline{157}$\partial_1.$\mdline{157} For the reverse inclusion, suppose we have a \mdline{157}$0$\mdline{157}-chain \mdline{157}$\sum_i n_i \sigma_i$\mdline{157} such that \mdline{157}$\sum_i n_i = 0.$\mdline{157} Pick some basepoint \mdline{157}$x_0 \in X,$\mdline{157} and for each \mdline{157}$i,$\mdline{157} pick a path \mdline{157}$\tau_i$\mdline{157} from \mdline{157}$x_0$\mdline{157} to \mdline{157}$x_i.$\mdline{157} Each \mdline{157}$\tau_i$\mdline{157} is a singular \mdline{157}$1$\mdline{157}-simplex, whose 0th face is \mdline{157}$\sigma_i$\mdline{157} and whose 1st face is \mdline{157}$\sigma_0.$\mdline{157} It follows that \mdline{157}$\partial \tau_i = \sigma_i - \sigma_0,$\mdline{157} and hence that \mdline{157}$\partial\qty(\sum_i n_i \tau_i) = \sum_i n_i \sigma_i - \sum_i n_i \sigma_0 = \sum_i n_i \sigma_i.$\mdline{157} \mdline{157}\textbf{QED}\mdline{157}%mdk

%mdk-data-line={159}
\mdline{159}\textbf{Proposition:}\mdline{159} If \mdline{159}$X$\mdline{159} is a point, then \mdline{159}$H_n(X) = 0$\mdline{159} for \mdline{159}$n > 0$\mdline{159} and \mdline{159}$H_0(X) = \Z.$\mdline{159}%mdk

%mdk-data-line={161}
\mdline{161}\emph{Proof:}\mdline{161} For each \mdline{161}$n,$\mdline{161} there is exactly one singular \mdline{161}$n$\mdline{161}-simplex in \mdline{161}$X.$\mdline{161} This means that all the chain groups are \mdline{161}$\Z,$\mdline{161} and the boundary maps are given by \mdline{161}$\partial_n = 0$\mdline{161} for \mdline{161}$n$\mdline{161} odd and \mdline{161}$\partial_n = \id$\mdline{161} for \mdline{161}$n$\mdline{161} even. \mdline{161}\textbf{QED}\mdline{161}%mdk

%mdk-data-line={163}
\subsection{\mdline{163}3.4.\hspace*{0.5em}\mdline{163}Reduced homology groups}\label{sec-reduced-homology-groups}%mdk%mdk

%mdk-data-line={165}
\noindent\mdline{165}\textbf{Definition:}\mdline{165} The \mdline{165}\textbf{\emph{reduced homology groups}}\mdline{165} \mdline{165}$\tilde{H}_n(X)$\mdline{165} of a nonempty space \mdline{165}$X$\mdline{165} are the homology groups of the augmented chain complex%mdk
\noindent\noindent\[%mdk-data-line={167}
\cdots \longrightarrow C_2(X)
\overset{\partial_2}{\longrightarrow} C_1(X)
\overset{\partial_1}{\longrightarrow} C_0(X)
\overset{\epsilon}{\longrightarrow} \Z
\longrightarrow 0.
\]%mdk
\noindent\mdline{173}(Check that this is indeed a chain complex!) Here we have \mdline{173}$\tilde{H}_n(X) = H_n(X)$\mdline{173} for \mdline{173}$n > 0,$\mdline{173} and \mdline{173}$H_0(X) \equiv \tilde{H}_0(X) \oplus \Z$\mdline{173} splits, with \mdline{173}$\Z$\mdline{173} coming from the path component of the basepoint. One can imagine that the extra \mdline{173}$\Z$\mdline{173} in the chain complex comes from the unique map of the \mdline{173}$(-1)$\mdline{173}-simplex (i.e. the empty set) into \mdline{173}$X.$\mdline{173}

%mdk-data-line={175}
\mdline{175}\textbf{Theorem:}\mdline{175} If \mdline{175}$X$\mdline{175} is path-connected, then there is a map \mdline{175}$\pi_1(X) \to H_1(X)$\mdline{175} that realizes \mdline{175}$H_1(X)$\mdline{175} as the abelianization of \mdline{175}$\pi_1(X).$\mdline{175}%mdk

%mdk-data-line={177}
\mdline{177}We won\mdline{177}'\mdline{177}t prove this for now, since it would be too much of a detour into homotopy.%mdk

%mdk-data-line={181}
\section{\mdline{181}4.\hspace*{0.5em}\mdline{181}Lecture 4 (2016-09-06)}\label{sec-lecture-4-2016-09-06}%mdk%mdk

%mdk-data-line={183}
\subsection{\mdline{183}4.1.\hspace*{0.5em}\mdline{183}Homotopy invariance of singular homology}\label{sec-homotopy-invariance-of-singular-homology}%mdk%mdk

%mdk-data-line={185}
\noindent\mdline{185}\textbf{Proposition:}\mdline{185} \mdline{185}$f: X \to Y$\mdline{185} (a continuous map) induces \mdline{185}$f_*: H_n(X) \to H_n(Y)$\mdline{185} (a homomorphism).%mdk

%mdk-data-line={187}
\mdline{187}\emph{Proof:}\mdline{187} Define \mdline{187}$f_\#: C_n(X) \to C_n(Y)$\mdline{187} to send a singular \mdline{187}$n$\mdline{187}-simplex \mdline{187}$\sigma$\mdline{187} in \mdline{187}$X$\mdline{187} to the singular \mdline{187}$n$\mdline{187}-simplex \mdline{187}$f \circ \sigma$\mdline{187} in \mdline{187}$Y,$\mdline{187} extending by linearity. We need to show that \mdline{187}$f_\#$\mdline{187} is a \mdline{187}\textbf{\emph{chain map}}\mdline{187}, that is, \mdline{187}$f_\# \partial = \partial f_\#.$\mdline{187} (This is equivalent to requiring that a ladder diagram commute.)%mdk

%mdk-data-line={189}
\mdline{189}\emph{(Straightforward verification omitted.)}\mdline{189} \mdline{189}\textbf{QED}\mdline{189}%mdk

%mdk-data-line={191}
\mdline{191}\textbf{Lemma:}\mdline{191} A chain map between chain complexes induces a homomorphism on homology.%mdk

%mdk-data-line={193}
\mdline{193}\emph{Proof:}\mdline{193} \mdline{193}$f_\#$\mdline{193} takes cycles to cycles. Indeed, if \mdline{193}$\partial\alpha = 0,$\mdline{193} then \mdline{193}$\partial f_\# \alpha = f_\# \partial \alpha = f_\# 0 = 0.$\mdline{193} Moreover, \mdline{193}$f_\#$\mdline{193} takes boundaries to boundaries, since \mdline{193}$f_\#\partial\beta = \partial f_\# \beta.$\mdline{193} \mdline{193}\textbf{QED}\mdline{193}%mdk

%mdk-data-line={195}
\mdline{195}\textbf{Properties:}\mdline{195}%mdk

%mdk-data-line={197}
\begin{enumerate}[noitemsep,topsep=\mdcompacttopsep]%mdk

%mdk-data-line={197}
\item\mdline{197}$(f \circ g)_* = f_* \circ g_*$\mdline{197}%mdk

%mdk-data-line={198}
\item\mdline{198}$(\id_X)_* = \id_{H_n(X)}$\mdline{198}%mdk
%mdk
\end{enumerate}%mdk

%mdk-data-line={200}
\noindent\mdline{200}This shows singular homology is a functor.%mdk

%mdk-data-line={202}
\mdline{202}\textbf{Theorem:}\mdline{202} If \mdline{202}$f,g: X \to Y$\mdline{202} are homotopic, then they induce the same homomorphism on homology.%mdk

%mdk-data-line={204}
\mdline{204}\emph{Corollary:}\mdline{204} If \mdline{204}$f: X \to Y$\mdline{204} is a homotopy equivalence, then \mdline{204}$f_*: H_n(X) \to H_n(Y)$\mdline{204} is an isomorphism for all \mdline{204}$n.$\mdline{204}%mdk

%mdk-data-line={206}
\mdline{206}\emph{Proof of Theorem:}\mdline{206} We first need to understand how to decompose \mdline{206}$\Delta^n \times I$\mdline{206} into \mdline{206}$(n+1)$\mdline{206}-simplices. We will omit some combinatorial details (see Hatcher for more), but if we regard \mdline{206}$\Delta^n \times I$\mdline{206} as having \mdline{206}$[v_0, \dots, v_n]$\mdline{206} on its bottom and \mdline{206}$[w_0, \dots, w_n]$\mdline{206} on top, then%mdk
\noindent\noindent\[%mdk-data-line={208}
[v_0, \dots, v_i, w_i, \dots, w_n]
\]%mdk
\noindent\mdline{210}is an \mdline{210}$(n+1)$\mdline{210}-simplex. The collection of these simplices for all \mdline{210}$i$\mdline{210} gives a decomposition of \mdline{210}$\Delta^n \times I$\mdline{210} into \mdline{210}$(n+1)$\mdline{210}-simplices.

%mdk-data-line={212}
\mdline{212}Now, given a homotopy \mdline{212}$F: X \times I \to Y$\mdline{212} between \mdline{212}$f = F_0$\mdline{212} and \mdline{212}$g = F_1,$\mdline{212} we define \mdline{212}\textquotedblleft{}prism operators\textquotedblright{}\mdline{212} \mdline{212}$P: C_n(X) \to C_{n+1}(Y)$\mdline{212} as follows:%mdk
\noindent\noindent\[%mdk-data-line={214}
P(\sigma) = \sum_{i=0}^n (-1)^i \eval{\qty[F \circ (\sigma \times \id_I)]}_{[v_0, \dots, v_i, w_i, \dots, w_n]}
\]%mdk
\noindent\mdline{217}We claim that \mdline{217}$\partial P = g_\# - f_\# - P\partial.$\mdline{217} Intuitively, we\mdline{217}'\mdline{217}re just saying that \mdline{217}\textquotedblleft{}the boundary of a prism consists of the top plus the bottom plus the sides,\textquotedblright{}\mdline{217} with some signs thrown in for good measure. Checking this is a straightforward calculation:
\noindent\noindent\[%mdk-data-line={219}
\begin{aligned}
\partial P(\sigma) &= \sum_{j \le i} (-1)^i (-1)^j
\eval{[F \circ (\sigma \times \id_I)]}_{
  [v_0, \dots, \hat{v}_j, \dots, v_i, w_i, \dots, w_n]} \\
&\phantom{{}={}} + \sum_{j \ge i} (-1)^i (-1)^{j+1}
\eval{[F \circ (\sigma \times \id_I)]}_{
  [v_0, \dots, v_i, w_i, \dots, \hat{w}_j, \dots, w_n]}
\end{aligned}
\]%mdk
\noindent\mdline{228}All of the \mdline{228}$i=j$\mdline{228} terms cancel, except for two: for \mdline{228}$i=j=0,$\mdline{228} we get
\noindent\noindent\[%mdk-data-line={230}
\eval{[F \circ (\sigma \times \id_I)]}_{
  [\hat{v}_0, w_0, \dots, w_n]} = g_\#(\sigma),
\]%mdk
\noindent\mdline{233}and for \mdline{233}$i=j=n,$\mdline{233} we get
\noindent\noindent\[%mdk-data-line={235}
-\eval{[F \circ (\sigma \times \id_I)]}_{
  [v_0, \dots, v_n, \hat{w}_n]} = -f_\#(\sigma).
\]%mdk
\noindent\mdline{238}The rest of the terms give precisely \mdline{238}$P(\partial\sigma).$\mdline{238}

%mdk-data-line={240}
\mdline{240}We now claim that this proves the theorem! If \mdline{240}$\alpha \in C_n(X)$\mdline{240} is a cycle, then%mdk
\noindent\noindent\[%mdk-data-line={242}
g_\#(\alpha) - f_\#(\alpha) = \partial P(\alpha).
\]%mdk
\noindent\mdline{244}Since \mdline{244}$g_\#(\alpha)$\mdline{244} and \mdline{244}$f_\#(\alpha)$\mdline{244} differ by a boundary, they represent the same homology class.

%mdk-data-line={246}
\mdline{246}\textbf{Remark:}\mdline{246} The identity \mdline{246}$\partial P + P \partial = g_\# - f_\#$\mdline{246} says that \mdline{246}$P$\mdline{246} is a \mdline{246}\textquotedblleft{}chain homotopy\textquotedblright{}\mdline{246} between \mdline{246}$g_\#$\mdline{246} and \mdline{246}$f_\#.$\mdline{246}%mdk

%mdk-data-line={248}
\mdline{248}\textbf{Remark:}\mdline{248} The exact same proof can be applied to reduced homologies, giving the same result.%mdk

%mdk-data-line={250}
\subsection{\mdline{250}4.2.\hspace*{0.5em}\mdline{250}Excision}\label{sec-excision}%mdk%mdk

%mdk-data-line={252}
\noindent\mdline{252}When \mdline{252}$A$\mdline{252} is a subspace of \mdline{252}$X,$\mdline{252} the homologies \mdline{252}$H_*(X),$\mdline{252} \mdline{252}$H_*(A),$\mdline{252} and \mdline{252}$H_*(X/A)$\mdline{252} are related by \mdline{252}\emph{exact sequences}\mdline{252}. A sequence of Abelian groups%mdk
\noindent\noindent\[%mdk-data-line={254}
\cdots \longrightarrow A_{n+1}
\overset{\alpha_{n+1}}{\longrightarrow} A_n
\overset{\alpha_{n}}{\longrightarrow} A_{n-1}
\longrightarrow \cdots
\]%mdk
\noindent\mdline{259}is said to be \mdline{259}\textbf{\emph{exact}}\mdline{259} if \mdline{259}$\ker\alpha_n = \im\alpha_{n+1}$\mdline{259} for all \mdline{259}$n.$\mdline{259} This means that the chain complex has trivial homology. For example,

%mdk-data-line={261}
\begin{itemize}[noitemsep,topsep=\mdcompacttopsep]%mdk

%mdk-data-line={261}
\item\mdline{261}If \mdline{261}$0 \longrightarrow A \overset{\alpha}{\longrightarrow} B$\mdline{261} is exact, then \mdline{261}$\alpha$\mdline{261} is injective.%mdk

%mdk-data-line={262}
\item\mdline{262}If \mdline{262}$A \overset{\alpha}{\longrightarrow} B \longrightarrow 0$\mdline{262} is exact, then \mdline{262}$\alpha$\mdline{262} is surjective.%mdk

%mdk-data-line={263}
\item\mdline{263}If \mdline{263}$0 \longrightarrow A \overset{\alpha}{\longrightarrow} B \longrightarrow 0$\mdline{263} is exact, then \mdline{263}$\alpha$\mdline{263} is an isomorphism.%mdk

%mdk-data-line={264}
\item\mdline{264}If \mdline{264}$0 \longrightarrow A \overset{\alpha}{\longrightarrow} B \overset{\beta}{\longrightarrow} C \longrightarrow 0$\mdline{264} is exact, then \mdline{264}$\alpha$\mdline{264} is injective, \mdline{264}$\beta$\mdline{264} is surjective, and \mdline{264}$\ker\beta = \im\alpha,$\mdline{264} so \mdline{264}$C = B/\im\alpha$\mdline{264}. Sequences of this form are called \mdline{264}\textbf{\emph{short exact sequences}}\mdline{264}.%mdk
%mdk
\end{itemize}%mdk

%mdk-data-line={266}
\noindent\mdline{266}\textbf{Theorem:}\mdline{266} If \mdline{266}$X$\mdline{266} is a space and \mdline{266}$A$\mdline{266} is a nonempty closed subspace that is a deformation retract of some neighborhood in \mdline{266}$X,$\mdline{266} then there is an exact sequence%mdk
\noindent\noindent\[%mdk-data-line={268}
\cdots \to \widetilde{H}_n(A)
\to \widetilde{H}_n(X)
\to \widetilde{H}_n(X/A)
\to \widetilde{H}_{n-1}(A)
\to \widetilde{H}_{n-1}(X)
\to \cdots
\to \widetilde{H}_0(X/A)
\to 0.
\]%mdk
\noindent\mdline{277}Here, the maps \mdline{277}$\widetilde{H}_n(A) \to \widetilde{H}_n(X)$\mdline{277} are induced by the inclusion map, and the maps \mdline{277}$\widetilde{H}_n(X) \to \widetilde{H}_n(X/A)$\mdline{277} are induced by the quotient map. The maps \mdline{277}$\widetilde{H}_n(X/A) \to \widetilde{H}_{n-1}(A)$\mdline{277} are non-obvious and will be constructed during the proof.

%mdk-data-line={279}
\mdline{279}\textbf{Remark:}\mdline{279} Pairs of spaces \mdline{279}$(X,A)$\mdline{279} satisfying these hypotheses are called \mdline{279}\textquotedblleft{}good pairs.\textquotedblright{}\mdline{279} (Really!)%mdk

%mdk-data-line={281}
\mdline{281}This theorem can be used to find, for example, the homology groups of \mdline{281}$S^n.$\mdline{281}%mdk

%mdk-data-line={285}
\section{\mdline{285}5.\hspace*{0.5em}\mdline{285}Lecture 05 (2016-09-08)}\label{sec-lecture-05-2016-09-08}%mdk%mdk

%mdk-data-line={287}
\noindent\mdline{287}\emph{(Missed some material on relative homology and the long exact homology sequence.)}\mdline{287}%mdk

%mdk-data-line={289}
\subsection{\mdline{289}5.1.\hspace*{0.5em}\mdline{289}Excision}\label{sec-excision}%mdk%mdk

%mdk-data-line={291}
\noindent\mdline{291}\textbf{Theorem:}\mdline{291} Given subspaces \mdline{291}$Z \subseteq A \subseteq X$\mdline{291} such that the closure of \mdline{291}$Z$\mdline{291} is contained in the interior of \mdline{291}$A,$\mdline{291} the inclusion of pairs \mdline{291}$(X-Z, A-Z) \hookrightarrow (X,A)$\mdline{291} induces isomorphisms \mdline{291}$H_n(X-Z, A-Z) \cong H_n(X,A)$\mdline{291} for all \mdline{291}$n.$\mdline{291}%mdk

%mdk-data-line={293}
\mdline{293}Equivalently, for subspaces \mdline{293}$A,B \subseteq X$\mdline{293} whose interiors cover \mdline{293}$X,$\mdline{293} then \mdline{293}$(B, A \cap B) \hookrightarrow (X,A)$\mdline{293} induces isomorphisms \mdline{293}$H_n(B, A \cap B) \cong H_n(X,A)$\mdline{293} for all \mdline{293}$n.$\mdline{293} To translate between these formulations, set \mdline{293}$B = X - Z$\mdline{293} or \mdline{293}$Z = X - B.$\mdline{293}%mdk

%mdk-data-line={295}
\mdline{295}\emph{Proof idea:}\mdline{295} Show that \mdline{295}$H_n(X)$\mdline{295} is actually giben by chains with \mdline{295}\textquotedblleft{}small images\textquotedblright{}\mdline{295} in \mdline{295}$X.$\mdline{295}%mdk

%mdk-data-line={297}
\mdline{297}Let \mdline{297}$\mathcal{U} = \{U_j\}$\mdline{297} be a collection of sets whose interiors cover \mdline{297}$X.$\mdline{297} Let \mdline{297}$C_n^\mathcal{U}(X)$\mdline{297} be the subgroup of \mdline{297}$C_n(X)$\mdline{297} consisting of chains \mdline{297}$\sum n_i \sigma_i$\mdline{297} such that the image of each \mdline{297}$\sigma_i$\mdline{297} is contained in one of the sets \mdline{297}$U_j \in \mathcal{U}.$\mdline{297} Note that the boundary maps \mdline{297}$\partial_n: C_n(X) \to C_{n-1}(X)$\mdline{297} take \mdline{297}$C_n^\mathcal{U}(X)$\mdline{297} to \mdline{297}$C_{n-1}^\mathcal{U}(X),$\mdline{297} so we can consider the homology \mdline{297}$H_n^\mathcal{U}(X)$\mdline{297} of this chain complex.%mdk

%mdk-data-line={299}
\mdline{299}\textbf{Proposition:}\mdline{299} The inclusion \mdline{299}$i: C_n^\mathcal{U}(X) \to C_n(X)$\mdline{299} is a chain homotopy equivalence. That is, there is a map \mdline{299}$p: C_n(X) \to C_n^\mathcal{U}(X)$\mdline{299} such that \mdline{299}$ip$\mdline{299} and \mdline{299}$pi$\mdline{299} are chain-homotopic to the identity. Hence, \mdline{299}$H_n^\mathcal{U}(X) \cong H_n(X).$\mdline{299}%mdk

%mdk-data-line={301}
\mdline{301}\textbf{Summary of Proof:}\mdline{301}%mdk

%mdk-data-line={303}
\begin{enumerate}%mdk

%mdk-data-line={303}
\item{}
%mdk-data-line={303}
\mdline{303}We first need to define \mdline{303}\emph{barycentric subdivision}\mdline{303}, a method of dividing simplices into smaller simplices.%mdk%mdk

%mdk-data-line={305}
\item{}
%mdk-data-line={305}
\mdline{305}We then create a subdivision operator on \mdline{305}\emph{linear chains}\mdline{305}, a restricted type of chain. We omit some details for now, but if \mdline{305}$Y \subseteq \R^m$\mdline{305} is a subset of Euclidean space, we want to consider the collection \mdline{305}$LC_n(Y)$\mdline{305} of linear maps \mdline{305}$\Delta^n \to Y.$\mdline{305} The subdivision operator \mdline{305}$S: LC_n(Y) \to LC_n(Y)$\mdline{305} should then break linear chains into smaller linear chains in such a way that \mdline{305}$\partial S = S \partial,$\mdline{305} i.e., such that \mdline{305}$S$\mdline{305} is a chain map. We also need to show that there\mdline{305}'\mdline{305}s a chain homotopy \mdline{305}$T: LC_n(Y) \to LC_{n+1}(Y)$\mdline{305} with \mdline{305}$\partial T + T \partial = \id - S.$\mdline{305}%mdk%mdk

%mdk-data-line={307}
\item{}
%mdk-data-line={307}
\mdline{307}We now pass to general chains. If \mdline{307}$\sigma \in C_n(X),$\mdline{307} define \mdline{307}$S(\sigma) = \sigma_\# S(\Delta^n),$\mdline{307} where if \mdline{307}$\sigma: \Delta^n \to X,$\mdline{307} then \mdline{307}$\sigma_\#: C_n(\Delta^n) \to C_n(X).$\mdline{307} We then define \mdline{307}$T: C_n(X) \to C_{n+1}(X)$\mdline{307} by \mdline{307}$T(\sigma) = \sigma_\# T(\Delta^n).$\mdline{307}%mdk%mdk

%mdk-data-line={309}
\item{}
%mdk-data-line={309}
\mdline{309}Iterate this construction. The amount of subdividing we need depends on the chain we started with.%mdk%mdk
%mdk
\end{enumerate}%mdk

%mdk-data-line={313}
\section{\mdline{313}6.\hspace*{0.5em}\mdline{313}Lecture 06 (2016-09-13)}\label{sec-lecture-06-2016-09-13}%mdk%mdk

%mdk-data-line={315}
\subsection{\mdline{315}6.1.\hspace*{0.5em}\mdline{315}Barycentric Subdivision}\label{sec-barycentric-subdivision}%mdk%mdk

%mdk-data-line={317}
\noindent\mdline{317}\textbf{Definition:}\mdline{317} The barycentric subdivision of \mdline{317}$\Delta^0$\mdline{317} (a point) is simply \mdline{317}$\Delta^0$\mdline{317}. The barycentric subdivision of \mdline{317}$\Delta^1$\mdline{317} (a line segment) is two line segments (drop a vertex in the middle). To subdivide \mdline{317}$\Delta^n$\mdline{317}, subdivide each of its faces \mdline{317}$\Delta^{n-1}$\mdline{317} inductively, and drop a new vertex in its barycenter. Then connect each division of each face to the new vertex to subdivide \mdline{317}$\Delta^n$\mdline{317} into \mdline{317}$(n+1)!$\mdline{317} pieces.%mdk

%mdk-data-line={319}
\mdline{319}\emph{(This is difficult to describe textually. See Hatcher's text for pictures.)}\mdline{319}%mdk

%mdk-data-line={321}
\mdline{321}\textbf{Fact:}\mdline{321} The diameter of each simplex in the barycentric subdivision of \mdline{321}$[v_0, \dots, v_n]$\mdline{321} is at most \mdline{321}$\frac{n}{n-1} \operatorname{diam}([v_0, \dots, v_n])$\mdline{321}. Thus, by repeatedly subdividing, we can make these diameters arbitrarily small.%mdk

%mdk-data-line={323}
\subsection{\mdline{323}6.2.\hspace*{0.5em}\mdline{323}Linear Chains}\label{sec-linear-chains}%mdk%mdk

%mdk-data-line={325}
\noindent\mdline{325}\textbf{Definition:}\mdline{325} Let \mdline{325}$Y$\mdline{325} be a subset of Euclidean space. We define \mdline{325}$LC_n(Y)$\mdline{325} as the free Abelian group generated by all linear maps \mdline{325}$\Delta^n \to Y$\mdline{325}.%mdk

%mdk-data-line={327}
\mdline{327}We now obtain a chain complex%mdk
\noindent\noindent\[%mdk-data-line={329}
\cdots \longrightarrow LC_n(Y)
\longrightarrow LC_{n-1}(Y)
\longrightarrow \cdots
\longrightarrow LC_{1}(Y)
\longrightarrow LC_{0}(Y)
\longrightarrow LC_{-1}(Y)
\]%mdk
\noindent\mdline{336}where, as before, \mdline{336}$LC_{-1}(Y)$\mdline{336} consists of linear maps from the empty simplex (i.e. \mdline{336}$[\varnothing]$\mdline{336}) to \mdline{336}$Y$\mdline{336}.

%mdk-data-line={338}
\mdline{338}\textbf{Definition:}\mdline{338} Given \mdline{338}$b \in Y$\mdline{338}, define \mdline{338}$b: LC_n(Y) \to LC_{n+1}(Y)$\mdline{338} by%mdk
\noindent\noindent\[%mdk-data-line={340}
b([w_0, \dots, w_n]) = [b, w_0, \dots, w_n]
\]%mdk
\noindent\mdline{342}where the RHS is obtained from the LHS by linear interpolation. (This is why we work with \mdline{342}$LC_n(Y)$\mdline{342} instead of \mdline{342}$C_n(Y)$\mdline{342}; linearity of the simplices guarantees that this interpolation can be done uniquely.) We call these maps \mdline{342}\textbf{\emph{cone operators}}\mdline{342} because they construct the \mdline{342}\textquotedblleft{}cone\textquotedblright{}\mdline{342} connecting the given simplex \mdline{342}$[w_0, \dots, w_n]$\mdline{342} to the new point \mdline{342}$b$\mdline{342}.

%mdk-data-line={344}
\mdline{344}It is a simple computation to check that%mdk
\noindent\noindent\[%mdk-data-line={346}
\partial b [w_0, \dots, w_n] = [w_0, \dots, w_n] - b \partial [w_0, \dots, w_n].
\]%mdk
\noindent\mdline{348}This means that \mdline{348}$b$\mdline{348} is a chain homotopy from \mdline{348}$\id$\mdline{348} to the zero map, since \mdline{348}$\partial b + b \partial = \id - 0$\mdline{348}.

%mdk-data-line={350}
\mdline{350}We now define \mdline{350}\textbf{\emph{subdivision homomorphisms}}\mdline{350} \mdline{350}$S: LC_n(Y) \to LC_n(Y)$\mdline{350} by induction. If \mdline{350}$\lambda: \Delta^n \to Y$\mdline{350}, and \mdline{350}$b_\lambda$\mdline{350} is the image of the barycenter of \mdline{350}$\Delta^n$\mdline{350} in \mdline{350}$Y$\mdline{350}, then%mdk
\noindent\noindent\[%mdk-data-line={352}
S(\lambda) = b_\lambda S(\partial \lambda).
\]%mdk
\noindent\mdline{354}We start this induction with \mdline{354}$S[\varnothing] = [\varnothing]$\mdline{354}, so that \mdline{354}$S = \id$\mdline{354} on \mdline{354}$n=-1$\mdline{354} and \mdline{354}$n=0$\mdline{354}.

%mdk-data-line={356}
\mdline{356}Observe that%mdk
\noindent\noindent\[%mdk-data-line={358}
\begin{aligned}
\partial S \lambda &= \partial b_\lambda (S \partial \lambda) \\
&= S \partial \lambda - b_\lambda \partial (S \partial \lambda) \\
&= S \partial \lambda - b_\lambda S \partial \partial \lambda.
\end{aligned}
\]%mdk
\noindent\mdline{364}This shows that \mdline{364}$\partial S = S \partial$\mdline{364}. We now want to build a chain homotopy \mdline{364}$T: LC_n(Y) \to LC_{n+1}(Y)$\mdline{364} between \mdline{364}$S$\mdline{364} and \mdline{364}$\id$\mdline{364}. Again, we work inductively: define \mdline{364}$T = 0$\mdline{364} for \mdline{364}$n=-1$\mdline{364}, and
\noindent\noindent\[%mdk-data-line={366}
T\lambda = b_\lambda(\lambda - T\partial\lambda)
\]%mdk
\noindent\mdline{368}for \mdline{368}$n \ge 0$\mdline{368}. We want to check that \mdline{368}$\partial T + T \partial = \id - S$\mdline{368} inductively.
\noindent\noindent\[%mdk-data-line={370}
\begin{aligned}
\partial T \lambda &= \partial b_\lambda (\lambda - T \partial \lambda) \\
&= \lambda - T\partial\lambda - b_\lambda \partial(\lambda - T\partial\lambda) \\
&= \lambda - T\partial\lambda - b_\lambda (S\partial\lambda + T\partial\partial\lambda) \\
&= \lambda - T\partial\lambda - S\lambda 
\end{aligned}
\]%mdk
\noindent\mdline{377}This is the desired result.

%mdk-data-line={379}
\subsection{\mdline{379}6.3.\hspace*{0.5em}\mdline{379}General Chains}\label{sec-general-chains}%mdk%mdk

%mdk-data-line={381}
\noindent\mdline{381}Now, we want to generalize to spaces \mdline{381}$X$\mdline{381} which are not subsets of Euclidean space. The key observation here is that \mdline{381}$\Delta^n$\mdline{381} is a subset of Euclidean space, so if we have a map \mdline{381}$\Delta^n \to X$\mdline{381}, we can subdivide \mdline{381}$\Delta^n$\mdline{381} and take compositions to obtain a subdivision of a singular \mdline{381}$n$\mdline{381}-simplex in \mdline{381}$X$\mdline{381}. We define \mdline{381}$S: C_n(X) \to C_n(X)$\mdline{381} by%mdk
\noindent\noindent\[%mdk-data-line={383}
S\sigma = \sigma_\# S(\Delta^n).
\]%mdk
\noindent\mdline{385}(Technically, we should write \mdline{385}$S(\id_{\Delta^n})$\mdline{385} instead of \mdline{385}$S(\Delta^n)$\mdline{385}, but this abuse of notation should not cause any confusion.) This means that \mdline{385}$S\sigma$\mdline{385} is a signed sum of restrictions of \mdline{385}$\sigma$\mdline{385} to \mdline{385}$n$\mdline{385}-simplices in the subdivision of \mdline{385}$\Delta^n$\mdline{385}. It is not hard to see that \mdline{385}$S$\mdline{385} is a chain map:
\noindent\noindent\[%mdk-data-line={387}
\begin{aligned}
\partial S \sigma &= \partial \sigma_\# S \Delta^n \\
&= \sigma_\# \partial S \Delta^n \\
&= \sigma_\# S \partial \Delta^n \\
&= \sigma_\# S \sum_{i=0}^n (-1)^i \Delta_i^n \\
&= \sum_{i=0}^n (-1)^i \sigma_\# S \Delta_i^n \\
&= \sum_{i=0}^n (-1)^i S\qty(\eval{\sigma}_{\Delta_i^n}) \\
&= S\qty(\sum_{i=0}^n (-1)^i \eval{\sigma}_{\Delta_i^n}) \\
&= S \partial \sigma
\end{aligned}
\]%mdk
\noindent\mdline{398}where we have written \mdline{398}$\Delta_i^n$\mdline{398} for the \mdline{398}$i$\mdline{398}th face of \mdline{398}$\Delta^n$\mdline{398}. We do the same thing with \mdline{398}$T$\mdline{398}, defining \mdline{398}$T: C_n(X) \to C_{n+1}(X)$\mdline{398} by
\noindent\noindent\[%mdk-data-line={400}
T\sigma = \sigma_\# T(\Delta^n).
\]%mdk
\noindent\mdline{402}A similar computation shows that \mdline{402}$\partial T + T \partial = \id - S$\mdline{402}.

%mdk-data-line={404}
\subsection{\mdline{404}6.4.\hspace*{0.5em}\mdline{404}Iteration}\label{sec-iteration}%mdk%mdk

%mdk-data-line={406}
\noindent\mdline{406}We now define%mdk
\noindent\noindent\[%mdk-data-line={408}
D_m = \sum_{i=0}^{m-1} TS^i
\]%mdk
\noindent\mdline{410}where \mdline{410}$S^i$\mdline{410} is the \mdline{410}$i$\mdline{410}-fold composition of \mdline{410}$S$\mdline{410}. We claim that \mdline{410}$D_m$\mdline{410} is a chain homotopy between \mdline{410}$\id$\mdline{410} and \mdline{410}$S^m$\mdline{410}, since
\noindent\noindent\[%mdk-data-line={412}
\begin{aligned}
\partial D_m + D_m \partial &= \sum_{i=0}^{m-1} \partial TS^i + TS^i \partial \\
&= \sum_{i=0}^{m-1} \partial TS^i + T\partial S^i \\
&= \sum_{i=0}^{m-1} (\partial T + T\partial) S^i \\
&= \sum_{i=0}^{m-1} (\id - S) S^i \\
&= \sum_{i=0}^{m-1} S^i - S^{i+1} \\
&= S^0 - S^m = \id - S^m.
\end{aligned}
\]%mdk
\noindent\mdline{422}We now claim that for any \mdline{422}$n$\mdline{422}-simplex \mdline{422}$\sigma: \Delta^n \to X$\mdline{422}, there exists \mdline{422}$m$\mdline{422} such that \mdline{422}$S^m(\sigma)$\mdline{422} lies in \mdline{422}$C_n^\mathcal{U}(X)$\mdline{422}. Indeed, we can always force the diameter of simplices in \mdline{422}$S^m(\Delta^n)$\mdline{422} smaller than the Lebesgue number of the cover \mdline{422}$\sigma^{-1}(\mathcal{U})$\mdline{422} by taking \mdline{422}$m$\mdline{422} sufficiently large. Henceforth, we will write \mdline{422}$m(\sigma)$\mdline{422} to denote the smallest \mdline{422}$m$\mdline{422} for which this holds.

%mdk-data-line={424}
\mdline{424}Define \mdline{424}$D: C_n(X) \to C_{n+1}(X)$\mdline{424} by \mdline{424}$D\sigma = D_{m(\sigma)}\sigma$\mdline{424}. We want a chain map \mdline{424}$\rho: C_n(X) \to C_n(X)$\mdline{424} whose image lies in \mdline{424}$C_n^\mathcal{U}(X)$\mdline{424} satisfying \mdline{424}$\partial D + D \partial = \id - \rho$\mdline{424}. Well, simply define \mdline{424}$\rho = \id - \partial D - D \partial$\mdline{424}. A simple computation shows that \mdline{424}$\partial \rho = \rho \partial$\mdline{424}. We need only check that \mdline{424}$\rho\sigma$\mdline{424} lies in \mdline{424}$C_n^\mathcal{U}(X)$\mdline{424}:%mdk
\noindent\noindent\[%mdk-data-line={426}
\begin{aligned}
\rho(\sigma) &= \sigma - \partial D \sigma - D \partial \sigma \\
&= \sigma - \partial D_{m(\sigma)} \sigma - D \partial \sigma \\
&= S^{m(\sigma)} \sigma  + D_{m(\sigma)} \partial \sigma - D \partial \sigma
\end{aligned}
\]%mdk
\noindent\mdline{432}Now, \mdline{432}$S^{m(\sigma)} \sigma$\mdline{432} definitely lies in \mdline{432}$C_n^\mathcal{U}(X)$\mdline{432}. Moreover, \mdline{432}$D_{m(\sigma)} \partial \sigma - D \partial \sigma$\mdline{432} is a linear combination of terms of the form \mdline{432}$D_{m(\sigma)} (\sigma_j) - D_{m(\sigma_j)} (\sigma_j)$\mdline{432} where \mdline{432}$\sigma_j$\mdline{432} is the \mdline{432}$j$\mdline{432}th face of \mdline{432}$\sigma$\mdline{432}. Since \mdline{432}$m(\sigma_j) \le m(\sigma)$\mdline{432}, we see that all these terms lie in \mdline{432}$C_n^\mathcal{U}(X)$\mdline{432}.

%mdk-data-line={434}
\mdline{434}Thus, we can view \mdline{434}$\rho$\mdline{434} as a map \mdline{434}$\rho: C_n(X) \to C_n^\mathcal{U}(X)$\mdline{434}. We see that \mdline{434}$\partial D + D \partial = \id - i\rho$\mdline{434} and \mdline{434}$\rho i = \id$\mdline{434}, as desired. \mdline{434}\textbf{QED}\mdline{434}%mdk

%mdk-data-line={436}
\mdline{436}\emph{Proof of Excision:}\mdline{436} Use the preceding proposition with \mdline{436}$\mathcal{U} = \{A, B\}$\mdline{436}. (All that work, and we\mdline{436}'\mdline{436}re only using a two-element cover!) Abusing notation, we denote \mdline{436}$C_n^\mathcal{U}(X) = C_n(A+B)$\mdline{436}. Observe that \mdline{436}$D$\mdline{436} and \mdline{436}$\rho$\mdline{436} take chains in \mdline{436}$A$\mdline{436} to chains in \mdline{436}$A$\mdline{436}. Thus, we obtain maps%mdk
\noindent\noindent\[%mdk-data-line={438}
C_n(A+B) / C_n(A) \overset{\rho}{\longrightarrow} C_n(X) / C_n(A)
\]%mdk
\noindent\mdline{440}satisfying some properties. (In particular, \mdline{440}$D$\mdline{440} is still a chain homotopy equivalence.) Now, \mdline{440}$H_n(B, A \cap B)$\mdline{440} is the homology of \mdline{440}$C_n(B) / C_n(A \cap B) = C_n(A+B) / C_n(A)$\mdline{440}. But the homology of \mdline{440}$C_n(X) / C_n(A)$\mdline{440} is precisely \mdline{440}$H_n(X, A)$\mdline{440}. Hence, \mdline{440}$H_n(B, A \cap B) \cong H_n(X, A)$\mdline{440}. \mdline{440}\textbf{QED}\mdline{440}%mdk


\end{document}
