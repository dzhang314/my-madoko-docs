\documentclass{article}
% generated by Madoko, version 1.0.1
%mdk-data-line={1}


\usepackage[heading-base={2},section-num={False},bib-label={True}]{madoko2}
\usepackage{bm}
\usepackage{physics}


\begin{document}



%mdk-data-line={8}
\newcommand{\R}{\mathbb{R}}
\renewcommand{\C}{\mathbb{C}}
\newcommand{\Rext}{{\R \cup \{\infty\}}}
\newcommand{\RExt}{{\R \cup \{-\infty, \infty\}}}
%mdk-data-line={14}
\mdxtitleblockstart{}
%mdk-data-line={14}
\mdxtitle{\mdline{14}Notes on Measure Theory}%mdk
\mdxauthorstart{}
%mdk-data-line={19}
\mdxauthorname{\mdline{19}David K. Zhang}%mdk
\mdxauthorend\mdtitleauthorrunning{}{}\mdxtitleblockend%mdk

%mdk-data-line={16}
\section{\mdline{16}1.\hspace*{0.5em}\mdline{16}Elementary Measure Theory}\label{sec-elementary-measure-theory}%mdk%mdk

%mdk-data-line={18}
\noindent\mdline{18}A \mdline{18}\textbf{\emph{measurable space}}\mdline{18} is an ordered pair \mdline{18}$(X, \Sigma)$\mdline{18} consisting of a set \mdline{18}$X$\mdline{18} and a collection \mdline{18}$\Sigma$\mdline{18} of subsets of \mdline{18}$X,$\mdline{18} satisfying the following axioms:%mdk

%mdk-data-line={20}
\begin{enumerate}[noitemsep,topsep=\mdcompacttopsep]%mdk

%mdk-data-line={20}
\item\mdline{20}$\varnothing \in \Sigma.$\mdline{20}%mdk

%mdk-data-line={21}
\item\mdline{21}$\Sigma$\mdline{21} is closed under taking complements.%mdk

%mdk-data-line={22}
\item\mdline{22}$\Sigma$\mdline{22} is closed under countable union and countable intersection.%mdk
%mdk
\end{enumerate}%mdk

%mdk-data-line={24}
\noindent\mdline{24}A collection \mdline{24}$\Sigma$\mdline{24} satisfying these properties is called a \mdline{24}\textbf{\emph{$\bm{\sigma}$-algebra}}\mdline{24} on \mdline{24}$X.$\mdline{24} The members of \mdline{24}$\Sigma$\mdline{24} are called \mdline{24}\textbf{\emph{measurable sets}}\mdline{24}. (This terminology is completely analogous to \mdline{24}\textquotedblleft{}topological space,\textquotedblright{}\mdline{24} \mdline{24}\textquotedblleft{}topology,\textquotedblright{}\mdline{24} and \mdline{24}\textquotedblleft{}open sets.\textquotedblright{}\mdline{24} Like in point-set topology, we will often notationally suppress the \mdline{24}$\sigma$\mdline{24}-algebra \mdline{24}$\Sigma$\mdline{24} when it is clear from context.)%mdk

%mdk-data-line={26}
\mdline{26}There is a standard way to turn a topological space \mdline{26}$(X, T)$\mdline{26} into a measurable space \mdline{26}$(X, \Sigma)$\mdline{26} by letting \mdline{26}$\Sigma$\mdline{26} be the closure of \mdline{26}$T$\mdline{26} under complements and countable union/intersection. The \mdline{26}$\sigma$\mdline{26}-algebra obtained in this fashion is called the \mdline{26}\textbf{\emph{Borel $\bm{\sigma}$-algebra}}\mdline{26} of the topological space \mdline{26}$(X, T),$\mdline{26} and the members of \mdline{26}$\Sigma$\mdline{26} are called \mdline{26}\textbf{\emph{Borel subsets}}\mdline{26} of \mdline{26}$X.$\mdline{26} When the set \mdline{26}$X$\mdline{26} has a usual topology (e.g. \mdline{26}$X = \R^n$\mdline{26} or \mdline{26}$X = \C^n$\mdline{26}) which is implicitly understood, we will speak of Borel subsets without further qualification.%mdk

%mdk-data-line={28}
\mdline{28}A function \mdline{28}$f: X \to Y$\mdline{28} between two measurable spaces is \mdline{28}\textbf{\emph{measurable}}\mdline{28} if the inverse image of every measurable set is measurable. Clearly, the identity function on any measurable space is measurable, and by the same argument as for continuous functions, the composition of two measurable functions is measurable. Thus, the collection of all measurable spaces forms a category, with measurable functions as morphisms.%mdk

%mdk-data-line={30}
\mdline{30}A \mdline{30}\textbf{\emph{measure}}\mdline{30} on a measure space \mdline{30}$(X, \Sigma)$\mdline{30} is a function \mdline{30}$\mu: \Sigma \to [0, \infty]$\mdline{30} satisfying the following axioms:%mdk

%mdk-data-line={32}
\begin{enumerate}[noitemsep,topsep=\mdcompacttopsep]%mdk

%mdk-data-line={32}
\item\mdline{32}$\mu(\varnothing) = 0.$\mdline{32}%mdk

%mdk-data-line={33}
\item\mdline{33}If \mdline{33}$\{A_i\}_{i=1}^\infty$\mdline{33} is a sequence of pairwise disjoint members of \mdline{33}$\Sigma$\mdline{33}, then
\noindent\noindent\[%mdk-data-line={35}
\mu\qty(\bigcup_{i=1}^\infty A_i) = \sum_{i=1}^\infty \mu(A_i).
\]%mdk%mdk
%mdk
\end{enumerate}%mdk

%mdk-data-line={38}
\noindent\mdline{38}If \mdline{38}$\mu$\mdline{38} also satisfies the additional axiom%mdk

%mdk-data-line={40}
\begin{enumerate}[noitemsep,topsep=\mdcompacttopsep,start=3]%mdk

%mdk-data-line={40}
\item\mdline{40}$\mu(X) = 1,$\mdline{40}%mdk
%mdk
\end{enumerate}%mdk

%mdk-data-line={42}
\noindent\mdline{42}Then we say that \mdline{42}$\mu$\mdline{42} is a \mdline{42}\textbf{\emph{probability measure}}\mdline{42}. The triple \mdline{42}$(X, \Sigma, \mu)$\mdline{42} is called a \mdline{42}\textbf{\emph{measure space}}\mdline{42}, or if \mdline{42}$\mu$\mdline{42} is a probability measure, a \mdline{42}\textbf{\emph{probability space}}\mdline{42}.%mdk

%mdk-data-line={44}
\mdhr{}%mdk

%mdk-data-line={46}
\noindent\mdline{46}Let \mdline{46}$X$\mdline{46} be an arbitrary topological space. A function \mdline{46}$f: X \to [-\infty, \infty]$\mdline{46} is \mdline{46}\textbf{\emph{lower semicontinuous}}\mdline{46} if \mdline{46}$f^{-1}([-\infty, c])$\mdline{46} is a closed subset of \mdline{46}$X$\mdline{46} for all \mdline{46}$c \in \R.$\mdline{46}%mdk

%mdk-data-line={48}
\mdline{48}\textbf{Proposition:}\mdline{48} If \mdline{48}$X$\mdline{48} is a metric space, then a function \mdline{48}$f: X \to [-\infty, \infty]$\mdline{48} is lower semicontinuous iff \mdline{48}$\displaystyle \liminf_{y \to x} f(y) \ge f(x)$\mdline{48} for all \mdline{48}$x \in X.$\mdline{48}%mdk

%mdk-data-line={50}
\mdline{50}\emph{Proof:}\mdline{50} Suppose \mdline{50}$f$\mdline{50} is lower semicontinuous, and let \mdline{50}$x \in X$\mdline{50} be given. It suffices to show that%mdk
\noindent\noindent\[%mdk-data-line={52}
\liminf_{n \to \infty} f(y_n) = \lim_{n \to \infty} \inf_{m \ge n} f(y_m) \ge f(x)
\]%mdk
\noindent\mdline{54}for all sequences \mdline{54}$\{y_n\}$\mdline{54} converging to \mdline{54}$x.$\mdline{54} Thus, suppose we are given such a sequence, and define
\noindent\noindent\[%mdk-data-line={56}
a_n = \inf_{m \ge n} f(y_m).
\]%mdk
\noindent\mdline{58}The sequence \mdline{58}$\{a_n\}$\mdline{58} increases monotonically, so to show that \mdline{58}$\lim_{n \to \infty} a_n \ge f(x),$\mdline{58} we simply need to show that \mdline{58}$\{a_n\}$\mdline{58} is not bounded above by any \mdline{58}$c < f(x).$\mdline{58} We proceed by contradiction: if \mdline{58}$\{a_n\}$\mdline{58} is bounded above by some \mdline{58}$c < f(x),$\mdline{58} then \mdline{58}$\{y_n\}$\mdline{58} lies inside the closed set \mdline{58}$f^{-1}([-\infty, c]).$\mdline{58} But by hypothesis, \mdline{58}$y_n$\mdline{58} converges to \mdline{58}$x,$\mdline{58} and because closed sets contain their limit points, we must have \mdline{58}$x \in f^{-1}([-\infty, c]).$\mdline{58} It follows that \mdline{58}$f(x) \le c < f(x);$\mdline{58} this is the desired contradiction.

%mdk-data-line={60}
\mdline{60}Conversely, suppose \mdline{60}$\liminf_{y \to x} f(y) \ge f(x)$\mdline{60} for all \mdline{60}$x \in X,$\mdline{60} and let \mdline{60}$c \in \R$\mdline{60} be given. We want to show that \mdline{60}$f^{-1}([-\infty, c])$\mdline{60} is a closed subset of \mdline{60}$X,$\mdline{60} so it suffices to show that it contains its limit points. Thus, suppose we have a sequence \mdline{60}$\{y_n\}$\mdline{60} lying inside \mdline{60}$f^{-1}([-\infty, c])$\mdline{60} which converges to a point \mdline{60}$x \in X.$\mdline{60} By hypothesis, \mdline{60}$\liminf_{n \to \infty} f(y_n) \ge f(x).$\mdline{60} Since the \mdline{60}$\liminf$\mdline{60} of a sequence upper-bounded by \mdline{60}$c$\mdline{60} is necessarily also \mdline{60}$\le c,$\mdline{60} we must have \mdline{60}$f(x) \le c$\mdline{60}, and \mdline{60}$x \in f^{-1}([-\infty, c]),$\mdline{60} as desired. \mdline{60}\textbf{QED}\mdline{60}%mdk

%mdk-data-line={62}
\mdline{62}If we are given an arbitrary function \mdline{62}$f: X \to [-\infty, \infty],$\mdline{62} we can construct a lower semicontinuous function \mdline{62}$f_\text{lsc},$\mdline{62} called the \mdline{62}\textbf{\emph{lower semicontinuous regularization}}\mdline{62} of \mdline{62}$f,$\mdline{62} as follows:%mdk
\noindent\noindent\[%mdk-data-line={64}
f_\text{lsc}(x) = \sup_{G \ni x \text{ open}} \qty[\inf_{y \in G} f(y)]
\]%mdk
\noindent\mdline{66}First, let us verify that this function is indeed lower semicontinuous.

%mdk-data-line={68}
\mdline{68}\textbf{Proposition:}\mdline{68} Let \mdline{68}$X$\mdline{68} be an arbitrary topological space and \mdline{68}$f: X \to [-\infty, \infty]$\mdline{68} an arbitrary function.%mdk

%mdk-data-line={70}
\mdline{70}\emph{Proof:}\mdline{70} Let \mdline{70}$c \in \R$\mdline{70} be given. We want to show that \mdline{70}$f^{-1}([\infty, c])$\mdline{70} is a closed subset of \mdline{70}$X.$\mdline{70}%mdk%mdk


\end{document}
